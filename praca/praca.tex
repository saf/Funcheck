\documentclass{pracamgr}

\usepackage[polish]{babel}
\usepackage[utf8]{inputenc}
\usepackage{t1enc}

\author{Sławomir Rudnicki}
\nralbumu{248277}

\title{Weryfikacja funkcyjności metod Javy}
\tytulang{Verifying that Java methods are functions}
\kierunek{Informatyka}

\opiekun{dr. Aleksego Schuberta\\
  Instytut Informatyki\\
  }
\date{Maj 2011}
\dziedzina{11.3 Informatyka}
\klasyfikacja{D. Software\\
  D.2 SOFTWARE ENGINEERING\\
  D.2.4. Software/Program Verification
}

\keywords{Java, purity, functions, verification, logic programming}

\newtheorem{defi}{Definicja}[section]

\begin{document}
\maketitle

\begin{abstract}
  Abstract %TODO
\end{abstract}

\tableofcontents
%\listoffigures
%\listoftables

\chapter*{Wprowadzenie}
\addcontentsline{toc}{chapter}{Wprowadzenie}

Programowanie funkcyjne jest jednym z~głównych paradygmatów
programowania używanych we współczesnej informatyce. Jego idea
oraz~główne zasady zostały sformułowana w~1977~roku przez Johna
Backusa~\cite{backus} i stanowiły podstawę dla wielu języków
programowania, takich jak szeroko używane obecnie OCaml, Standard ML
oraz wprowadzony niedawno język $F\sharp$.

Jednym z głównych założeń programowania funkcyjnego jest oparcie go o
procedury zachowujące się tak jak funkcje matematyczne. Po pierwsze
więc, wynik wywołania takiej procedury zależy wyłącznie od wartości
parametrów, z którymi została wywołana. Po drugie, wykonanie procedury
skutkuje wyłącznie wyliczeniem wartości funkcji, a poza tym nie wpływa
na sposób wyliczania innych procedur: procedury funkcyjne nie mają
\emph{efektów ubocznych}.

Powyższe własności procedur można przenieść na języki imperatywne,
takie jak Java. Metody klas Javy zachowujące te własności będziemy
nazywać metodami \emph{czysto funkcyjnymi}, lub krótko
\emph{funkcyjnymi} (ang. \emph{pure}). Własność czystej
funkcyjności metod i klas zostanie ściśle zdefiniowana w rozdziale
\ref{property-definitions} niniejszej pracy.

Pojęciem związanym z funkcyjnością jest \emph{niemutowalność} obiektów
i klas Javy.  Obiekt $X$ jest niemutowalny, jeżeli żaden inny obiekt
nie może zaobserwować dwóch różnych stanów obiektu $X$. Oznacza to, że
stan obiektu może zmieniać się wyłącznie w czasie wykonania jego
konstruktora. Stan obiektu należy tu rozumieć jednak w sposób szerszy
niż tylko wartości jego pól. W przypadku niektórych klas, pojęcie
reprezentacji obiektu -- intuicyjnie rozumianej jako obszar pamięci,
którego wartość definiuje stan obiektu -- musi zostać rozszerzone, by
obejmować również stan tych obiektów, do których główny obiekt
utrzymuje referencje.

Dzięki własnościom niemutowalności obiektów oraz funkcyjności ich
metod uzyskujemy pożądane cechy obiektów i całych systemów, takie jak
wyeliminowanie zagrożeń związanych z wielowątkowym wykonaniem programu
oraz bezpieczeństwo w niezaufanym środowisku.

W ramach niniejszej pracy zaprojektowano i zaimplementowano narzędzie
\emph{Funcheck}, które pozwala na definiowanie i weryfikację własności
funkcyjności metod oraz niemodyfikowalności obiektów. Przyjęte w pracy
podejście do weryfikacji własności programów jest oparte o analizę
statyczną kodu

\chapter{Podstawowe pojęcia}\label{r:pojecia}



\appendix

\chapter{Załącznik 1}
\chapter{Załącznik 2} %TODO


\addcontentsline{toc}{chapter}{Bibliografia}
\bibliography{biblio.bib}{}
\bibliographystyle{plain}

\end{document}
