\documentclass{pracamgr}

\usepackage[polish]{babel}
\usepackage[utf8]{inputenc}
\usepackage{t1enc}

\author{Sławomir Rudnicki}
\nralbumu{248277}

\title{Weryfikacja funkcyjności metod Javy}
\tytulang{Verifying that Java methods are functions}
\kierunek{Informatyka}

\opiekun{dr. Aleksego Schuberta\\
  Instytut Informatyki\\
  }
\date{Maj 2011}
\dziedzina{11.3 Informatyka}
\klasyfikacja{D. Software\\
  D.2 SOFTWARE ENGINEERING\\
  D.2.4. Software/Program Verification
}

\keywords{Java, purity, functions, verification, logic programming}

\newtheorem{defi}{Definicja}[section]

\begin{document}
\maketitle

\begin{abstract}
  Abstract %TODO
\end{abstract}

\tableofcontents
%\listoffigures
%\listoftables

\chapter*{Wprowadzenie}

Programowanie funkcyjne jest jednym z~głównych paradygmatów
programowania używanych we współczesnej informatyce. Jego idea
oraz~główne zasady zostały sformułowana w~1977~roku przez Johna
Backusa~\cite{backus} i stanowiły podstawę dla wielu języków
programowania, takich jak szeroko używane obecnie OCaml, Standard ML
oraz wprowadzony niedawno język $F\sharp$.

Jednym z głównych założeń programowania funkcyjnego jest oparcie go o
procedury zachowujące się tak jak funkcje matematyczne. Po pierwsze
więc, wynik wywołania takiej procedury zależy wyłącznie od wartości
parametrów, z którymi została wywołana. Po drugie, wykonanie procedury
skutkuje wyłącznie wyliczeniem wartości funkcji, a poza tym nie wpływa
na sposób wyliczania innych procedur: procedury funkcyjne nie mają
\emph{efektów ubocznych}.

Powyższe własności procedur można przenieść na języki imperatywne,
takie jak Java. Metody klas Javy zachowujące te własności będziemy
nazywać metodami \emph{czysto funkcyjnymi}, lub krótko
\emph{funkcyjnymi} (ang. \emph{pure}). Własność czystej
funkcyjności metod i klas zostanie ściśle zdefiniowana w rozdziale
\ref{purity} niniejszej pracy.

Pojęciem związanym z funkcyjnością jest \emph{niemutowalność} obiektów
i klas Javy.  Obiekt $X$ jest niemutowalny, jeżeli żaden inny obiekt
nie może zaobserwować dwóch różnych stanów obiektu $X$. Oznacza to, że
stan obiektu może zmieniać się wyłącznie w czasie wykonania jego
konstruktora. Stan obiektu należy tu rozumieć jednak w sposób szerszy
niż tylko wartości jego pól. W przypadku niektórych klas, pojęcie
reprezentacji obiektu -- intuicyjnie rozumianej jako obszar pamięci,
którego wartość definiuje stan obiektu -- musi zostać rozszerzone, by
obejmować również stan tych obiektów, do których główny obiekt
utrzymuje referencje.

Dzięki własnościom niemutowalności obiektów oraz funkcyjności ich
metod uzyskujemy pożądane cechy obiektów i całych systemów, takie jak
wyeliminowanie zagrożeń związanych z wielowątkowym wykonaniem programu
oraz bezpieczeństwo w niezaufanym środowisku. Przykładowo, dwie metody
czysto funkcyjne mogą zostać bezpiecznie wykonane współbieżnie (w~
przeplocie), ponieważ brak efektów ubocznych zapewnia, że żadna z nich
nie wpływa na wynik drugiej.

\section*{Cel pracy}

W ramach niniejszej pracy magisterskiej zaprojektowano i
zaimplementowano narzędzie \emph{Funcheck}, które pozwala na
definiowanie i weryfikację własności funkcyjności metod oraz
niemodyfikowalności obiektów. 

Przyjęte w pracy podejście do weryfikacji własności programów jest
oparte o analizę statyczną kodu i dowodzenie przy pomocy stworzonego
na potrzeby weryfikacji własności funkcyjności systemu typów. Ów
system typów bazuje na systemie typów przedstawionym w pracy
dr. dr. Jacka Chrząszcza i Aleksego Schuberta \emph{Functional
  Java}. % \cite{functional} ??

Praca została stworzona w ramach seminarium \emph{Niezawodność systemów 
współbieżnych i obiektowych} w roku akademickim $2010/11$.


\section*{Struktura pracy}

Praca składa się z \ref{conclusion} rozdziałów:
\begin{description}
\item[Rozdział \ref{purity}] definiuje własności programów
  weryfikowane przez narzędzie \emph{Funcheck},
\item[Rozdział \ref{project}] zawiera omówienie metod i narzędzi
  zastosowanych w projekcie i implementacji narzędzia \emph{Funcheck}, 
\item[Rozdział \ref{type-system}] zawiera definicję typów i reguł
  typowania używanych do weryfikacji funkcyjności metod Javy.
\item[Rozdział \ref{verification-results}] zawiera przykłady działania
  stworzonego narzędzia: programy, które są poprawnie weryfikowane
  przez narzędzie \emph{Funcheck}, jak i te, które ze względu na
  niepełność opracowanego systemu typów nie mogą zostać poprawnie
  zweryfikowane.
\item[Rozdział \ref{conclusion}] stanowi podsumowanie pracy. 
\end{description}

\chapter{Weryfikowane własności}
\label{purity}

Celem tego rozdziału jest szczegółowe określenie własności
funkcyjności i niemutowalności dla elementów języka Java. W tym celu
najpierw zostaną wprowadzone definicje podstawowych pojęć związanych z
semantyką Javy, a następnie przy ich użyciu zostaną zdefiniowane 
omawiane własności. 

\section{Podstawowe pojęcia}

\chapter{Projekt narzędzia \emph{Funcheck}}
\label{project}

W niniejszym rozdziale zostanie przedstawiony ogólny projekt narzędzia
\emph{Funcheck}. Zostaną opisane poszczególne fazy przetwarzania
danych o weryfikowanym systemie oraz narzędzia, które zostały użyte w 
czasie tworzenia programu. 

\section{Architektura narzędzia}



\section{Weryfikacja oparta o systemy typów}

Jedno z podejść do zagadnienia weryfikacji własności programów polega
na opracowaniu odpowiedniego systemu typów charakteryzujących
syntaktyczne elementy języka. W ten sposób problem sprawdzania
żądanych własności systemu informatycznego można sprowadzić do
problemu przypisania typów. To podejście było wykorzystane na przykład
do zapewniania bezpieczeństwa i żywotności w systemach współbieżnych
przez Naoki Kobayashiego \cite{tb-concurrent}. Analogicznej metody ten
sam autor użył również do weryfikacji pewności uwierzytelnienia w
protokołach kryptograficznych \cite{tb-cryptographic}.

Zostały również opracowane systemy typów dla weryfikacji własności
omawianych w niniejszej pracy. System typów dla niemutowalności
obiektów stworzyli C. Haack, E. Poll, J. Schäfer i A. Schubert
\cite{immutability}. System typów dla weryfikacji funkcyjności metod
powstaje natomiast na Wydziale MIM UW, przy udziale dr. dr. Jacka
Chrząszcza i Aleksego Schuberta; niniejsza praca jest w dużej mierze
oparta o nieopublikowany szkic ich pracy \emph{Functional Java}.

\chapter{System typów dla zapewniania funkcyjności}
\label{type-system}

\chapter{Przykłady działania programu}
\label{verification-results}

\chapter{Podsumowanie}
\label{conclusion}

\appendix

\chapter{Załącznik 1} %TODO out?

\addcontentsline{toc}{chapter}{Bibliografia}
\bibliography{biblio.bib}{}
\bibliographystyle{plain}

\end{document}
