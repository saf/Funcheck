\documentclass{beamer}

\usetheme{Warsaw}
\usefonttheme[onlylarge]{structurebold}
\setbeamerfont*{frametitle}{size=\normalsize,series=\bfseries}
\setbeamertemplate{navigation symbols}{}

\usepackage[polish]{babel}
\usepackage[utf8]{inputenc}
\usepackage{times}
\usepackage[T1]{fontenc}
\usepackage{listings}
\usepackage{graphicx}
\usepackage{wrapfig}
\usepackage{enumerate}

\definecolor{ugreen}{rgb}{0, 0.5, 0}
\definecolor{lgreen}{rgb}{0.85, 1, 0.85}
\definecolor{lstback}{RGB}{235 240 250}

\setbeamercolor{gr}{fg=white, bg=ugreen}
\setbeamercolor{lgr}{fg=black, bg=lgreen}

\title{JSR 308 i weryfikacja adnotacji Javy}

\author{Sławomir Rudnicki} 

\institute{Niezawodność systemów współbieżnych i obiektowych}

\date{27 października 2010}

\lstset{language=Java, showstringspaces=false, backgroundcolor=\color{lstback}, 
        emph={@NonNull, @SuppressWarnings, @Deprecated}, 
        emphstyle=\color{red}}

\begin{document}

\begin{frame}
  \titlepage
\end{frame}
\begin{frame}
  \tableofcontents
\end{frame}

\section{Wprowadzenie}

\begin{frame}{Sfera zainteresowań na dziś}
\begin{itemize}
\item[$\rightarrow$] weryfikacja własności elementów programu:
\begin{itemize}
\item klas, 
\item metod, 
\item obiektów, 
\item pól obiektów...
\end{itemize}
\end{itemize}
\end{frame}

\begin{frame}{Sfera zainteresowań na dziś}
\begin{itemize}
\item<1->[$\rightarrow$] własności definiowane przy pomocy \textbf{adnotacji}
\begin{center}
\lstinputlisting{code/annot-example.java}
\end{center}
\item<2->[$\rightarrow$] weryfikacja będzie odbywać się \textbf{statycznie} -- analiza kodu źródłowego
\end{itemize}
\end{frame}

\section{Adnotacje}
\subsection{Historia}
\begin{frame}{Adnotacje w Java 5}
\begin{itemize}
\item<1-> Adnotacje zostały wprowadzone w Javie 5 na podstawie propozycji
  JSR 175 z 2002 roku.
\item<2-> Cel: 
\begin{itemize}
\item umożliwienie wprowadzania metadanych do kodu źródłowego
\end{itemize}
\item<3-> Postać adnotacji: 
\begin{itemize}
\item \color{red} @NonNull 
\item \color{red} @Deprecated
\end{itemize}
\end{itemize}
\end{frame}

\begin{frame}{Adnotowane elementy}
  \structure{Klasy}  
  \lstinputlisting{code/annot-class.java}
  \pause
  \structure{Metody}
  \lstinputlisting{code/annot-method.java}
\end{frame}

  

\subsection{Własne adnotacje}
\subsection{JSR 308}
\begin{frame}
\end{frame}

\end{document}
